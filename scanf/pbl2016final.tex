%%%%%%%%%%%%%%%%%%%%%%%%%%%%%%%%%%%%%%%%%%%%%%%%%%%%%%%%%%%%%%%%
% 情報工学PBL:システム開発演習
% グループ作業報告書テンプレート
% 作成日:2016-12-15
% 作成者:佐藤 和彦
%%%%%%%%%%%%%%%%%%%%%%%%%%%%%%%%%%%%%%%%%%%%%%%%%%%%%%%%%%%%%%%%
\documentclass[a4j]{jarticle}
\usepackage{ascmac}
\usepackage{fancybox}
\usepackage[dvips]{graphics}
\voffset=-23pt
\hoffset=-18pt
\headheight=10pt
\headsep=5pt
\topmargin=5pt
\textheight=248mm
\textwidth=160mm
\columnsep=8mm
\parindent=8pt
\parskip=0pt
%-------------------------------------------------
\begin{document}\pagestyle{empty}
\begin{center}
{\Large\bf 平成28年度情報工学PBL:システム開発演習 活動報告レポート}\par
%-------------------------------------------------
\vspace*{5mm}


%%%%%%%%%%%%%%%%%%%%%%%%%%%%%%%%%%%%%%%%%%%%
% 以下、コメントが付いている部分を追記する %
%%%%%%%%%%%%%%%%%%%%%%%%%%%%%%%%%%%%%%%%%%%%


%%%%%%%%%%%%%%%%%%%%%%%%%%%%%%%%%%%%%%%%%%%%%%%%%%%%%%%%%%%%%%%%
% ↓グループ番号&学生氏名
%%%%%%%%%%%%%%%%%%%%%%%%%%%%%%%%%%%%%%%%%%%%%%%%%%%%%%%%%%%%%%%%
{\large グループ番号:GP01\hspace*{1em} %←グループ番号
{j15024031 奥 龍司}
}\\
\end{center}

%%%%%%%%%%%%%%%%%%%%%%%%%%%%%%%%%%%%%%%%%%%%%%%%%%%%%%%%%%%%%%%%
\section{作業報告}
%---------------------------------------------------------------
%グループとしてではなく、自分自身が行った作業の中での工夫点や
%苦労した点などについてまとめなさい
%---------------------------------------------------------------
%%%%%%%%%%%%%%%%%%%%%%%%%%%%%%%%%%%%%%%%%%%%%%%%%%%%%%%%%%%%%%%%

%=================================================
\subsection{グループ内での自分の役割}
% 自分の担当した作業などについて簡潔にまとめなさい。
%=================================================

%記入例
%私は、地図上でのグラフ表示とそのアニメーション変化を担当しました。
%具体的には・・・
全国の地方ごとの中心地点の座標を国土地理院のサイトやGoogleMapを用いて探した。実装上で全国の各地点が表示されている状態からその座標に近づけると、その地域の詳しい天気が多量に表示されるようにするためである。



%=================================================
\subsection{最も工夫(苦労)した点}
% 自分が担当した作業内容について一番アピールできる点を
% 簡潔にまとめなさい
%=================================================

%記入例
%グラフを地図上に表示する機能を担当しましたが、各グラフを○○させる
%ことができたことが一番の工夫です。これは・・・
ある地方に近づこうとしたら別の地方が表示され、今見ている地方の天気が全く表示されない状態を改善することが苦労した。



%=================================================
\subsection{やり残した点}
% 自分の担当作業において本当はこうしたかった、
% こうするはずだったなど、やり残したことについて
% もしあるならばここで述べなさい。
% 予定通りに作業が進み、やり残しが特にないならば
% その旨を書いて下さい。
%=================================================

%記入例
%予定していた作業はすべて、計画通りに達成できました。
上記の苦労した点で全ての地方が改善されたわけではなく、表示されない地方があった。その他、ある地方の詳しい天気を表示しようとして、特定の距離から近すぎたら別の場所が表示されるという状態のままだった。


%%%%%%%%%%%%%%%%%%%%%%%%%%%%%%%%%%%%%%%%%%%%%%%%%%%%%%%%%%%%%%%%
\section{作業履歴}
%---------------------------------------------------------------
% 週ごとの作業内容についてまとめなさい。
% 記載されていない作業は「やっていない」ことになり採点対象外に
% なります。
% 演習時間外に行った作業についても、どの週で何を行ったかを
% 思い出せる範囲で履歴に全て含めなさい。
% やりかけて中止した作業についても中止した旨を添えて記入しなさい。
%---------------------------------------------------------------
%%%%%%%%%%%%%%%%%%%%%%%%%%%%%%%%%%%%%%%%%%%%%%%%%%%%%%%%%%%%%%%%
\begin{center}
\renewcommand{\arraystretch}{1.4}
\begin{tabular}{|p{25mm}|p{125mm}|}\hline
%=================================================

%↓↓↓↓↓↓↓ 以下を編集 ↓↓↓↓↓↓↓↓↓↓

第06週(11/17) & 企画案の発言 \\\cline{2-2}    % ←「作業内容」を書き換える ※以下同じ
			 & どのように完成物を見せるのかの提案 \\\hline
  
第07週(11/24) & ある程度動いていたので、どのように見やすく、カッコよく作るかの提案をした \\\hline    
              %& 作業内容2 \\\cline{2-2}
              
第08週(12/1) & 自分のPCで設計物がちゃんと動作できるように試みた \\\hline
              %& $\vdots$ \\\hline          % ←この行は消してください
第09週(12/8) & Javaの勉強 \\\cline{2-2}
			& 次週の中間発表に向けて、発表物の操作の確認・おさらい \\\hline
              %& $\vdots$ \\\hline          % ←この行は消してください
第10週(12/15) & 発表物の操作 \\\hline
              %& $\vdots$ \\\hline          % ←この行は消してください
第11週(1/5) & Javaの勉強 \\\cline {2-2}
			& 気圧や風向きなどの利用する可能性のある情報の収集 \\\hline 
第12週(01/12) & 天気の表示データ量が多い状態だと致命的に動作が重いので、Viewpointの位置に近い地方の天気を表示するようにした\\\hline
             % & $\vdots$ \\\hline          % ←この行は消してください
第13週(01/19) & 雛形から残っている地震関連の不必要なデータを消去した \\\hline
             % & $\vdots$ \\\hline          % ←この行は消してください
第14週(01/26) & 中間発表と同様に発表物の操作担当だったので操作方法の一通りのおさらい\\\hline
第15週(02/02) & 発表で使用するヘッドセットが発表物の映像を映さないというまさかのアクシデントが起きたので、GP01のメンバーと協力してなんとか発表できる映像をスクリーンに映せるような行動をとった。 \\\hline 

%↑↑↑↑↑↑↑ ここまで ↑↑↑↑↑↑↑↑↑↑

%=================================================
\end{tabular}
\end{center}

%%%%%%%%%%%%%%%%%%%%%%%%%%%%%%%%%%%%%%%%%%%%%%%%%%%%%%%%%%%%%%%%
\section{PBL演習で習得したこと}
%---------------------------------------------------------------
%この授業科目は情報工学科の学習目標の以下の項目に対応している.
%  ・人[自己啓発]自己を啓発して学習する習慣を身につける。
%  ・人[チームワーク力]他者と共同して仕事を進める能力を身につける.
%  ・技術者[段取り力]論理的に計画を立案し合理的に段取りを設定して課題を解決する能力を身につける.
%  ・情報技術者[情報システム]情報システムの基礎知識と構築・運用能力を身につける.
%
%  PBL演習を通して上記の目標に対してどのような成長が自分自身に
%  あったかについてそれぞれ簡潔にまとめなさい。
%---------------------------------------------------------------
%%%%%%%%%%%%%%%%%%%%%%%%%%%%%%%%%%%%%%%%%%%%%%%%%%%%%%%%%%%%%%%%

\subsection*{人[自己啓発]}
Javaの基礎知識を少しながら身につけることができた。

\subsection*{人[チームワーク力]}
元々話し合いでは発言する方であり、メンバーも友人が多かったのでチームワークにはさほど困らなかったが集団で仕事が進めるという経験を得ることができた。

\subsection*{技術者[段取り力]}
最終的に形になるようなものを出来はしたが改善すべき課題も残ってしまったのでこちらの力はおざなりになってしまったと思う。

\subsection*{情報技術者[情報システム]}
Netbeansの使い方やVR空間の構築が、なんとなくという程度ではあるが理解できるようになった。


%%%%%%%%%%%%%%%%%%%%%%%%%%%%%%%%%%%%%%%%%%%%%%%%%%%%%%%%%%%%%%%%
\section{自己評価}
%---------------------------------------------------------------
% 自分の成果について自己評価を手短にまとめなさい。
% 教員からのコメントや最終発表での他グループの成果を見ての比較でも良い。
% 自分たちの成果に対する客観的評価が前提ですが、自分達の成果を
% 否定的にとらえ過ぎないようにして下さい。比較して自分達が優れ
% ていると感じた点などを重視しなさい。
% 
% 記入内容についての例1:
% ○○先生から「□□機能は△△すればもっと使いやすくなる」との
% コメントを頂いたが、実際は作成したシステムでは指摘のような
% 機能として実装をしていた。最終発表ではその点について十分に
% 説明できなかったので発表の構成を少し見直すべきだった。
% 
% 記入内容についての例2:
% 他のグループの成果を見て、自分たちのプログラミングスキルが
% まだまだ不足していると感じたが、作成した機能の操作性に関して
% 見ると、その点にこだわった成果がそれなりに出て、同じような
% 機能を作成したグループと比べて我々のほうが使いやすくできている
% と思った。
%---------------------------------------------------------------
%%%%%%%%%%%%%%%%%%%%%%%%%%%%%%%%%%%%%%%%%%%%%%%%%%%%%%%%%%%%%%%%

%簡潔に数行程度(10行以内)でまとめること。
どのようなものを作るのかという最初の企画の話し合いでは何を作るのか、どういったコンセプトのものを作るかの意見や提案を積極的に出した。\\ 
しかし開発になると基本的にリーダーにほとんど任せっきりになってしまった状態は否めなかった。Javaについてはある程度は勉強をしたがあまり理解していない状態で演習が進んだためリーダーに大部分を頼ってしまった。


%***************************************************************
% ↑ここまでが4ページ以内に収まるようにまとめなさい
%***************************************************************
\end{document}
