%==========================================================================
%Template File
%   Copyright (C) 2006-2009                                
%           by Shinya Watanabe(sin@csse.muroran-it.ac.jp) 
%==========================================================================
\documentclass[a4j,titlepage]{jarticle}
\usepackage{sty/programing_report}
\usepackage{cases}
\usepackage{sty/jquote}
\usepackage{sty/eclbkbox}
\usepackage{sty/itembkbx}
\usepackage{sty/emathC}
\usepackage{graphicx}
\begin{document}

%--------------------
%以下に実験レポートのタイトル,自分のクラス名,学籍番号,氏名,提出日を書く.
%--------------------

%%\title{レポートタイトル}を記述する.
\title{第3回数値解析 レポート}

%%\author{クラス名}{学籍番号}{氏名}を記述する.
\author{15024031}{奥 龍司}

%%\date{提出する年月日}を記述する.
\date{2017年2月9日}
\maketitle

%--------------------
%以下から本文を開始する.
%--------------------


\section{プログラムソース}
 以下に今回の課題で作ったプログラムのソースを記述する。
 \begin{breakitembox}[l]{スプライン補間} \small
\begin{verbatim}

\end{verbatim}
\end{breakitembox}




\section{プログラムの実行}
プログラムの実行結果を以下に記す。
\begin{breakitembox}[l]{スプライン補間} \small
\begin{verbatim}


\end{verbatim}
\end{breakitembox}

\section{図}
以下に実行結果によって求まった補間曲線の図を表記する


\end{document}
