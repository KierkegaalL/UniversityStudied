%==========================================================================
%Template File
%   Copyright (C) 2006-2009                                
%           by Shinya Watanabe(sin@csse.muroran-it.ac.jp) 
%==========================================================================
\documentclass[a4j,titlepage]{jarticle}
\usepackage{sty/programing_report}
\usepackage{cases}
\usepackage{sty/jquote}
\usepackage{sty/eclbkbox}
\usepackage{sty/itembkbx}
\usepackage{sty/emathC}
\usepackage{graphicx}
\begin{document}

%--------------------
%以下に実験レポートのタイトル,自分のクラス名,学籍番号,氏名,提出日を書く.
%--------------------

%%\title{レポートタイトル}を記述する.
\title{第3回数値解析 レポート}

%%\author{クラス名}{学籍番号}{氏名}を記述する.
\author{15024031}{奥 龍司}

%%\date{提出する年月日}を記述する.
\date{2017年2月9日}
\maketitle

%--------------------
%以下から本文を開始する.
%--------------------


\section{プログラムソース(h=0.5)}
 以下に今回の課題で作った、h=0.5の場合のプログラムのソースを記述する。
 
 \begin{breakitembox}[l]{オイラー法} \small
\begin{verbatim}

/********************************************
名前: Eular.c
機能: オイラー法を用いて微分方程式の解を求める
作者: 15024031 奥 龍司
特記事項:特になし
********************************************/
#include<stdio.h>
#include<math.h>
#define h 0.5


double dy(double x, double y);


int main(void){
    double x, x_max, y;
    
    y = 0.5;
    
    x = 0;
    x_max = 10;
    
    while(x < x_max)
    {
        printf("%lf \n",y);
        x  = x + h;
        y  = y + dy(x,y) * h;
    }
    return 0;
}

double dy(double x, double y){ //求める微分方程式
    return 2 * y/(1 + x);
}


\end{verbatim}
\end{breakitembox}


 \begin{breakitembox}[l]{オイラー法} \small
\begin{verbatim}

/********************************************
名前: Eular2.c
機能: 修正オイラー法を用いて微分方程式の解を求める
作者: 15024031 奥 龍司
特記事項:特になし
********************************************/
#include<stdio.h>
#include<math.h>
#define h 0.5


double dy(double x, double y);


int main(void){
    double x, x_max, y, y1;
    
    y = 0.5;
    y1= 0;
    x = 0;
    x_max = 10;
    
    while(x < x_max)
    {
        printf("%lf \n",y);
        x  = x + h;
        y1 = y + dy(x,y) * h;
        y  = y + (y + y1)* h/2.0;
    }
    return 0;
}

double dy(double x, double y){ //求める微分方程式
    return 2 * y/(1 + x);
}



\end{verbatim}
\end{breakitembox}

 
\begin{breakitembox}[l]{ルンゲ・クッタ法} \small
\begin{verbatim}

/********************************************
名前: RungeKutta.c
機能: ルンゲ・クッタ法を用いて微分方程式の解を求める
作者: 15024031 奥 龍司
特記事項:特になし
********************************************/
#include<stdio.h>
#include<math.h>
#define h 0.5

double dy(double x, double y);

int main(void){
    double x, x_max, y, k1, k2, k3, k4;
    
    y = 0.5;
    
    x = 0;
    x_max = 10;
    
    while(x < x_max)
    {
        printf("%lf \n",y);
        x  = x + h;
        k1 = h * dy(x, y);
        k2 = h * dy(x + h/2, y + k1/2);
        k3 = h * dy(x + h/2, y + k2/2);
        k4 = h * dy(x + h, y + k3);
        y  = y + (k1 + 2*k2 + 2*k3 + k4)/6;
    }
    
  return 0;
}

double dy(double x, double y){//求める微分方程式
    return 2 * y/(1 + x);
}

\end{verbatim}
\end{breakitembox}




\section{プログラムの実行(h=0.5)}
h=0.5の場合のプログラムの実行結果を以下に記す。なお、xの値の範囲は0から10までをこのレポートに記すとする。
\begin{breakitembox}[l]{オイラー法} \small
\begin{verbatim}
0.500000 
0.833333 
1.250000 
1.750000 
2.333333 
3.000000 
3.750000 
4.583333 
5.500000 
6.500000 
7.583333 
8.750000 
10.000000 
11.333333 
12.750000 
14.250000 
15.833333 
17.500000 
19.250000 
21.083333
\end{verbatim}
\end{breakitembox}

\begin{breakitembox}[l]{修正オイラー法} \small
\begin{verbatim}

\end{verbatim}
\end{breakitembox}

\begin{breakitembox}[l]{ルンゲ・クッタ法} \small
\begin{verbatim}
0.500000 
0.888605 
1.388309 
1.999088 
2.720934 
3.553843 
4.497811 
5.552837 
6.718922 
7.996063 
9.384262 
10.883517 
12.493829 
14.215198 
16.047622 
17.991104 
20.045641 
22.211235 
24.487885 
\end{verbatim}
\end{breakitembox}


\section{プログラムソース(h=0.01)}
 以下に今回の課題で作った、h=0.01の場合のプログラムのソースを記述する。
 
\begin{breakitembox}[l]{オイラー法} \small
\begin{verbatim}

/********************************************
名前: Eular.c
機能: オイラー法を用いて微分方程式の解を求める
作者: 15024031 奥 龍司
特記事項:特になし
********************************************/
#include<stdio.h>
#include<math.h>
#define h 0.01


double dy(double x, double y);


int main(void){
    double x, x_max, y;
    
    y = 0.5;
    
    x = 0;
    x_max = 1;
    
    while(x < x_max)
    {
        printf("%lf \n",y);
        x  = x + h;
        y  = y + dy(x,y) * h;
    }
    return 0;
}

double dy(double x, double y){ //求める微分方程式
    return 2 * y/(1 + x);
}


\end{verbatim}
\end{breakitembox}

\begin{breakitembox}[l]{修正オイラー法} \small
\begin{verbatim}

/********************************************
名前: Eular2.c
機能: 修正オイラー法を用いて微分方程式の解を求める
作者: 15024031 奥 龍司
特記事項:特になし
********************************************/

\end{verbatim}
\end{breakitembox}


\begin{breakitembox}[l]{ルンゲ・クッタ法} \small
\begin{verbatim}

/********************************************
名前: RungeKutta.c
機能: ルンゲ・クッタ法を用いて微分方程式の解を求める
作者: 15024031 奥 龍司
特記事項:特になし
********************************************/
#include<stdio.h>
#include<math.h>
#define h 0.01

double dy(double x, double y);

int main(void){
    double x, x_max, y, k1, k2, k3, k4;
    
    y = 0.5;
    
    x = 0;
    x_max = 1;
    
    while(x < x_max)
    {
        printf("%lf \n",y);
        x  = x + h;
        k1 = h * dy(x, y);
        k2 = h * dy(x + h/2, y + k1/2);
        k3 = h * dy(x + h/2, y + k2/2);
        k4 = h * dy(x + h, y + k3);
        y  = y + (k1 + 2*k2 + 2*k3 + k4)/6;
    }
    
  return 0;
}

double dy(double x, double y){ //求める微分方程式
    return 2 * y/(1 + x);
}
\end{verbatim}
\end{breakitembox}



\section{プログラムの実行(h=0.01)}
h=0.01の場合のプログラムの実行結果を以下に記す。なお、xの値の範囲は0から1までをこのレポートに記すとする。
\begin{breakitembox}[l]{オイラー法} \small
\begin{verbatim}
0.500000 
0.509901 
0.519899 
0.529994 
0.540186 
0.550476 
0.560862 
0.571345 
0.581926 
0.592603 
0.603378 
0.614250 
0.625218 
0.636284 
0.647447 
0.658707 
0.670064 
0.681518 
0.693069 
0.704718 
0.716463 
0.728305 
0.740245 
0.752281 
0.764415 
0.776645 
0.788973 
0.801398 
0.813920 
0.826539 
0.839255 
0.852068 
0.864978 
0.877985 
0.891089 
0.904290 
0.917589 
0.930984 
0.944477 
0.958066 
0.971753 
0.985537 
0.999418 
1.013395 
1.027470 
1.041642 
1.055911 
1.070278 
1.084741 
1.099301 
1.113958 
1.128713 
1.143564 
1.158513 
1.173559 
1.188701 
1.203941 
1.219278 
1.234712 
1.250243 
1.265871 
1.281596 
1.297418 
1.313337 
1.329354 
1.345467 
1.361677 
1.377985 
1.394389 
1.410891 
1.427490 
1.444186 
1.460978 
1.477868 
1.494855 
1.511939 
1.529121 
1.546399 
1.563774 
1.581246 
1.598816 
1.616482 
1.634246 
1.652106 
1.670064 
1.688119 
1.706271 
1.724520 
1.742865 
1.761308 
1.779849 
1.798486 
1.817220 
1.836051 
1.854980 
1.874005 
1.893128 
1.912347 
1.931664 
1.951077 
\end{verbatim}
\end{breakitembox}

\begin{breakitembox}[l]{修正オイラー法} \small
\begin{verbatim}

\end{verbatim}
\end{breakitembox}

\begin{breakitembox}[l]{ルンゲ・クッタ法} \small
\begin{verbatim}
0.500000 
0.509950 
0.519998 
0.530144 
0.540388 
0.550730 
0.561170 
0.571709 
0.582345 
0.593079 
0.603911 
0.614842 
0.625870 
0.636996 
0.648221 
0.659543 
0.670964 
0.682482 
0.694099 
0.705813 
0.717626 
0.729536 
0.741545 
0.753652 
0.765856 
0.778159 
0.790560 
0.803059 
0.815655 
0.828350 
0.841143 
0.854034 
0.867023 
0.880110 
0.893295 
0.906578 
0.919959 
0.933438 
0.947015 
0.960690 
0.974463 
0.988334 
1.002304 
1.016371 
1.030536 
1.044800 
1.059161 
1.073620 
1.088178 
1.102833 
1.117587 
1.132438 
1.147388 
1.162435 
1.177581 
1.192824 
1.208166 
1.223606 
1.239143 
1.254779 
1.270513 
1.286344 
1.302274 
1.318302 
1.334428 
1.350652 
1.366974 
1.383394 
1.399912 
1.416528 
1.433242 
1.450054 
1.466964 
1.483972 
1.501078 
1.518283 
1.535585 
1.552985 
1.570483 
1.588080 
1.605774 
1.623566 
1.641457 
1.659445 
1.677532 
1.695716 
1.713999 
1.732379 
1.750858 
1.769434 
1.788109 
1.806882 
1.825752 
1.844721 
1.863788 
1.882953 
1.902215 
1.921576 
1.941035 
1.960592
\end{verbatim}
\end{breakitembox}

%\section{考察}


\end{document}
 
