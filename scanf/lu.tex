%==========================================================================
%Template File
%   Copyright (C) 2006-2009                                
%           by Shinya Watanabe(sin@csse.muroran-it.ac.jp) 
%==========================================================================
\documentclass[a4j,titlepage]{jarticle}
\usepackage{sty/programing_report}
\usepackage{cases}
\usepackage{sty/jquote}
\usepackage{sty/eclbkbox}
\usepackage{sty/itembkbx}
\usepackage{sty/emathC}
\usepackage{graphicx}
\begin{document}

%--------------------
%以下に実験レポートのタイトル,自分のクラス名,学籍番号,氏名,提出日を書く.
%--------------------

%%\title{レポートタイトル}を記述する.
\title{第2回数値解析 レポート}

%%\author{クラス名}{学籍番号}{氏名}を記述する.
\author{15024031}{奥 龍司}

%%\date{提出する年月日}を記述する.
\date{2016年12月1日}
\maketitle

%--------------------
%以下から本文を開始する.
%--------------------


\section{プログラムソース}
 以下に今回の課題で作ったプログラムのソースを記述する。
 
\begin{breakitembox}[l]{LU分解} \small
\begin{verbatim}

/********************************************
名前: LU.c
機能: LU分解を用いて逆行列を求める
作者: 15024031 奥 龍司
特記事項:特になし
********************************************/
#include<stdio.h>
#include<stdlib.h>
#include<math.h>

#define N 4

	int i, j, k, x;
    double sum;
    double L[N][N] = {{1,0,0,0}, {0,1,0,0}, {0,0,1,0}, {0,0,0,1}};
    double U[N][N] = {{0,0,0,0}, {0,0,0,0}, {0,0,0,0}, {0,0,0,0}};
    double BeforeA[N][N]={{4,-6,-10,-2},{-6,8,21,-1}, {-10,21,-20,23},{-2,-1,23,-18}};
    double invA[N][N]={{4,-6,-10,-2},{-6,8,21,-1}, {-10,21,-20,23},{-2,-1,23,-18}};
    double AfterA[N][N];
    double y[N][N];
    double b[N][N] = {{1,0,0,0}, {0,1,0,0}, {0,0,1,0}, {0,0,0,1}};
    double h[N][N];

    
    /* 元々の行列の表示 */
    printf("求める行列Aはこちら\n");
    
    for(i=0; i<N; i++)
    {
        for(j=0; j<N; j++)
        {
            printf("%12.4f", BeforeA[i][j]);
        }
        printf("\n");
    }
    
    
    /* LU分解による計算 */
    for(j=0; j<N; j++)
    {
        U[0][j] = BeforeA[0][j];
    }
    for(j=1; j<N; j++)
    {
        L[j][0] = BeforeA[j][0] / U[0][0];
    }
    for(k=1; k<N; k++)
    {
        for(j=k; j<N; j++)
        {
            for(i=0; i<k; i++)
            {
                BeforeA[j][k] -= L[j][i] * U[i][k];
            }
        }
        
        U[k][k] = BeforeA[k][k];
        
        for(j=k+1; j<N; j++)
        {
            U[k][j] = BeforeA[k][j];
            for(i=0; i<k; i++)
            {
                U[k][j] -= L[k][i] * U[i][j];
            }
        }
        for(j=k+1; j<N; j++)
        {
            L[j][k] = BeforeA[j][k] / U[k][k];
        }
    }
    
    printf("下三角行列L \n");
    
    for(i=0; i<N; i++)
    {
        for(j=0; j<N; j++)
        {
            printf("%12.4f", L[i][j]);
        }
        printf("\n");
    }
    
    printf("上三角行列U \n");
    
    for(i=0; i<N; i++)
    {
        for(j=0; j<N; j++)
        {
            printf("%12.4f", U[i][j]);
        }
        printf("\n");
    }

    
    /* 前進代入 */
    for(i=0; i<N; i++)
    {
        for(j=0; j<N; j++)
        {
            y[j][i] = b[j][i];
            for(k=0; k<j; k++)
            {
                y[j][i] -= L[j][k] * y[k][i];
            }
        }
    }
    
    printf("ベクトルy \n");
    
    for(i=0; i<N; i++)
    {
        for(j=0; j<N; j++)
        {
            printf("%12.4f", y[i][j]);
        }
        printf("\n");
    }
    
    
    /* 後退代入 */
    for(x=0; x<N; x++)
    {
        for(k=1; k<=N; k++)
        {
            i = N - k;
            sum = y[i][x];
            
            for(j=i+1; j<N; j++)
            {
                sum -= U[i][j] * AfterA[j][x];
            }
            
            AfterA[i][x] = sum / U[i][i];
        }
    }
    
    
    /* データの出力 */
    printf("A の逆行列\n");
    
    for(i=0; i<N; i++)
    {
        for(j=0; j<N; j++)
        {
            printf("%12.4f", AfterA[i][j]);
        }
    printf("\n");
    }


    /*検算*/
    printf("検算の結果\n");
    
    for(i=0; i<N; i++)
    {
        for(j=0; j<N; j++)
        {
            for(k=0; k<N; k++)
            {
                h[i][j] += invA[i][k] * AfterA[k][j];
            }
           	printf("%12.4f",h[i][j]);
        }
    printf("\n");
    }
    
    
    return 0;
}
\end{verbatim}
\end{breakitembox}


\section{プログラムの実行}
プログラムの実行結果を以下に記す。
\begin{breakitembox}[l]{LU分解} \small
\begin{verbatim}
okuryuji-no-MacBook-Pro:数値解析 Ryuji$ gcc -o LU LU.c
okuryuji-no-MacBook-Pro:数値解析 Ryuji$ ./LU
求める行列Aはこちら
      4.0000     -6.0000    -10.0000     -2.0000
     -6.0000      8.0000     21.0000     -1.0000
    -10.0000     21.0000    -20.0000     23.0000
     -2.0000     -1.0000     23.0000    -18.0000
下三角行列L 
      1.0000      0.0000      0.0000      0.0000
     -1.5000      1.0000      0.0000      0.0000
     -2.5000     -6.0000      1.0000      0.0000
     -0.5000      4.0000      0.6667      1.0000
上三角行列U 
      4.0000     -6.0000    -10.0000     -2.0000
      0.0000     -1.0000      6.0000     -4.0000
      0.0000      0.0000     -9.0000     -6.0000
      0.0000      0.0000      0.0000      1.0000
ベクトルy 
      1.0000      0.0000      0.0000      0.0000
      1.5000      1.0000      0.0000      0.0000
     11.5000      6.0000      1.0000      0.0000
    -13.1667     -8.0000     -0.6667      1.0000
A の逆行列
    156.6667     96.1667      7.5000    -13.1667
     96.1667     59.0000      4.6667     -8.0000
      7.5000      4.6667      0.3333     -0.6667
    -13.1667     -8.0000     -0.6667      1.0000
検算の結果
	  1.0000      0.0000      0.0000      0.0000
     -0.0000      1.0000      0.0000      0.0000
     -0.0000     -0.0000      1.0000     -0.0000
      0.0000      0.0000      0.0000      1.0000
okuryuji-no-MacBook-Pro:数値解析 Ryuji$
\end{verbatim}
\end{breakitembox}

\section{考察}
 LU分解の時点でBeforeAの値が変わってしまったため検算の時に単位行列にならなかったため、新しく検算用の求める行列のinvAの値を作った。

\end{document}
